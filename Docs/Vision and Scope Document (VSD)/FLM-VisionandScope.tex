\documentclass[12pt]{article}
\usepackage[utf8]{inputenc}
\usepackage{graphicx}
\usepackage[a4paper,width=150mm,top=25mm,bottom=25mm]{geometry}
\usepackage{float}

\title{
{\includegraphics[width=4.5cm, height=4.5cm]{Figs/logo-uit-300x248.png}
}
\\
{Vision and Scope document (SVD) \\ Football League Management project}
}



\author{Hoang Tan, Tien Dat, Duy Ngo \\ Supervised by Lecture: Nguyen Thi Thanh Truc }

\begin{document}

\maketitle
\tableofcontents
\section{Problem Statement}
\subsection{Project Background}
The management of football leagues involves numerous administrative tasks, including club management, fixture scheduling, result tracking, ranking calculations, finance management, and communication with clubs. Traditionally, these tasks have been performed manually, leading to inefficiencies, errors, and difficulties in maintaining accurate and up-to-date records.

\subsection{ Problem Description}
The existing manual processes for managing football leagues pose several challenges and limitations. These include:
\begin{itemize}
    \item Tedious Administrative Tasks: The manual nature of tasks such as club management, fixture scheduling, and result tracking requires significant time and effort from league administrators. This often leads to delays, errors, and inconsistencies in the management of league operations.

    \item Inefficient Communication: Communication between league administrators and clubs is often fragmented, relying on various channels such as emails, phone calls, and physical documents. This hampers effective and timely information dissemination, leading to miscommunication and confusion among stakeholders.

    \item Lack of Real-time Updates: The absence of a centralized system results in a lack of real-time updates on fixtures, results, and league standings. Clubs and supporters struggle to stay informed about the latest information, hindering their engagement with the league.

    \item Data Discrepancies and Inaccuracies: Manual data entry and record-keeping increase the likelihood of errors and discrepancies in club profiles, fixture details, and financial records. This compromises the accuracy and reliability of league information, impacting decision-making processes.

    \item Limited Financial Visibility: The absence of a dedicated system for managing league finances makes it challenging to track income, expenses, and prize distributions accurately. This leads to inefficiencies in financial management and potential disputes among clubs.
\end{itemize}
\subsection{Stakeholders}
The Football League Management App project involves various stakeholders with different roles and interests. Identifying and understanding these stakeholders is crucial for the successful development and implementation of the application. The key stakeholders involved in the project include:
\begin{itemize}
    \item League Administrators: These individuals or organizations are responsible for overseeing the operations of the football league. They require a robust and efficient system to manage all aspects of the league, including club management, fixture scheduling, result tracking, ranking calculations, and financial management. League administrators need the app to streamline their administrative tasks, improve decision-making, and enhance overall league management efficiency.

    \item Clubs: The participating clubs are integral to the football league. They require a platform that allows them to manage their own information effectively. This includes player rosters, game schedules, performance metrics, and club finances. Clubs need the app to simplify club management tasks, provide real-time updates, and facilitate communication with league administrators and other clubs.

    \item Players: Football players associated with the participating clubs have a vested interest in the app. They need access to relevant information, such as game schedules, training sessions, team announcements, and performance statistics. The app should enable players to stay informed, monitor their progress, and engage with their teams and the league.

    \item Supporters: Fans and supporters of the football league play a vital role in its success. They have a strong interest in accessing up-to-date information on fixtures, results, league standings, and news related to their favorite clubs and players. The app should cater to supporters' needs by providing a user-friendly interface, engaging content, and features that foster fan engagement.

    \item Project Team: The development team consists of designers, developers, testers, project managers, and other professionals responsible for creating and delivering the Football League Management App. The project team has a direct interest in developing a high-quality application that meets the needs and expectations of the stakeholders. They are responsible for ensuring the app's functionality, performance, security, and user experience.

    \item Sponsors and Investors: Individuals, organizations, or entities providing financial support for the development and implementation of the app are essential stakeholders. Sponsors and investors have a vested interest in the success of the project and expect a return on their investment. They may provide guidance, resources, or requirements that need to be considered during the project's execution.

    \item Regulatory Bodies: Depending on the region and governing bodies associated with the football league, regulatory bodies may have specific requirements and guidelines that need to be adhered to. These stakeholders could include national or regional football associations, sports governing bodies, or legal entities. Compliance with their regulations and standards is essential for the app's acceptance and use within the league.
\end{itemize}
Understanding the needs, expectations, and perspectives of these stakeholders is critical for successfully defining the scope, features, and functionalities of the Football League Management App. Continuous communication, collaboration, and feedback from the stakeholders throughout the project lifecycle will ensure the app meets their requirements and contributes to the overall success of the football league.
\subsection{ User }
The Football League Management App caters to various users who interact with the system and benefit from its functionalities. Understanding the specific needs and requirements of these users is essential for designing a user-centric application. The primary users of the app are:
\begin{itemize}
    \item League Administrators: These users are responsible for managing the football league at an administrative level. They require comprehensive tools and features to efficiently handle club management, fixture scheduling, result tracking, ranking calculations, and financial management. The app should provide a user-friendly interface with intuitive controls, streamlined workflows, and robust reporting capabilities to support the administrators in making informed decisions and effectively managing the league.

    \item Club Managers: Club managers, appointed by participating clubs, are responsible for managing their respective clubs within the league. They need access to features that allow them to update club information, manage player rosters, communicate with league administrators, schedule training sessions, and track their club's financial transactions. The app should empower club managers with a dedicated portal where they can conveniently perform these tasks and access relevant information.

    \item Players: The app is also used by football players associated with the participating clubs. Players require access to features that provide them with up-to-date information about their team's schedule, training sessions, match results, and performance statistics. The app should offer personalized dashboards for players, allowing them to track their own performance, view team-related updates, and communicate with fellow players and club managers.

    \item Supporters and Fans: Supporters and fans of the football league play a crucial role in driving engagement and fostering a sense of community. These users seek access to fixtures, match results, league standings, news, and updates related to their favorite clubs and players. The app should provide a dedicated section where supporters can access this information, engage in discussions, share their enthusiasm, and stay connected with the league and its activities.

    \item Referees and Match Officials: Referees and match officials involved in the league require a platform to access relevant information about scheduled fixtures, venue details, match rules, and any updates or changes. The app should provide a dedicated interface for referees and officials to view their assigned matches, submit reports, communicate with league administrators, and access necessary documentation.

    \item Administrative Staff: Apart from league administrators, there may be additional administrative staff involved in managing the football league. These users may include personnel responsible for handling registrations, processing financial transactions, maintaining documentation, and coordinating with clubs and other stakeholders. The app should provide specific features and privileges tailored to their roles and responsibilities to facilitate smooth administrative operations.

    \item Technical Support Staff: Technical support staff or helpdesk personnel may be required to provide assistance to users of the app. These individuals should have access to support tools, resources, and documentation within the app to effectively address user queries, troubleshoot issues, and ensure a seamless user experience.
\end{itemize}
The Football League Management App aims to cater to the needs of these diverse user groups by providing a user-friendly, intuitive, and feature-rich interface that meets their specific requirements. By considering the unique perspectives and expectations of each user category, the app can facilitate efficient league management, enhance communication, and promote engagement among all stakeholders.

\subsection{Risks}

During the development and implementation of the Football League Management App, it's important to identify and mitigate potential risks that could impact the project's success. Understanding these risks allows for proactive planning and risk management strategies. The following are some potential risks associated with the project:
\begin{itemize}
    \item Technical Risks: There is a risk of encountering technical challenges during the development of the app. This could include compatibility issues with different operating systems and devices, scalability concerns as the user base grows, integration difficulties with existing systems or third-party services, and potential security vulnerabilities. To mitigate these risks, thorough testing, code reviews, and adherence to industry best practices will be implemented. Additionally, continuous monitoring and proactive maintenance will help ensure the app remains stable and secure.

    \item User Acceptance: There is a risk that the app may not meet the expectations or needs of the stakeholders, including league administrators, clubs, players, and supporters. This could result from poor user experience, inadequate functionality, or a lack of alignment with the specific requirements of the football league. To address this risk, extensive user research and feedback gathering will be conducted throughout the development process. Regular usability testing and iterative design improvements will be implemented to ensure the app is intuitive, user-friendly, and aligned with the stakeholders' expectations.

    \item Data Security and Privacy: As the app will handle sensitive information, such as personal player details and financial transactions, there is a risk of data breaches or unauthorized access. To mitigate this risk, robust security measures will be implemented, including encryption of data in transit and at rest, access controls, and regular security audits. Compliance with relevant data protection regulations and industry standards will be ensured to maintain the privacy and integrity of user data.

    \item Time and Resource Constraints: There is a risk of delays or resource limitations impacting the project's timeline and scope. This could be due to unforeseen technical challenges, changes in requirements, or insufficient availability of skilled resources. To mitigate this risk, a comprehensive project plan will be established with realistic timelines and resource allocations. Regular monitoring, effective communication, and proactive issue resolution will be key to managing time and resource constraints.

    \item Stakeholder Collaboration: Effective collaboration and communication with stakeholders, including league administrators, clubs, and players, are crucial for the success of the project. There is a risk of miscommunication, conflicting requirements, or lack of stakeholder engagement, which could impact the project's direction and outcomes. To mitigate this risk, regular stakeholder meetings, feedback sessions, and transparent communication channels will be established. A project management approach that encourages active participation and involvement of stakeholders will help ensure their needs are adequately addressed.

    \item External Dependencies: The project may have dependencies on external factors or third-party services, such as API integrations or infrastructure providers. There is a risk of delays or disruptions if these dependencies are not managed effectively. To mitigate this risk, thorough evaluation and selection of reliable partners and services will be conducted. Contingency plans and alternative solutions will be in place to mitigate any potential impact from external dependencies.
\end{itemize}
By identifying these potential risks early in the project and implementing appropriate risk mitigation strategies, the development team can proactively address challenges and ensure a successful delivery of the Football League Management App. Regular risk assessments and continuous monitoring throughout the project lifecycle will help identify new risks and allow for timely mitigation actions.

\subsection{Assumptions}
To ensure effective planning and decision-making during the development and implementation of the Football League Management App, it is important to make certain assumptions about the project. These assumptions serve as foundational beliefs and expectations that guide the project's scope, timeline, and deliverables. The following are key assumptions made for this project:
\begin{itemize}
    \item Availability of Sufficient Resources: It is assumed that the necessary resources, including skilled developers, designers, and project managers, will be available to execute the project successfully. Adequate hardware, software, and infrastructure resources will also be provided to support the development and deployment of the app.

    \item Stakeholder Collaboration and Engagement: It is assumed that stakeholders, including league administrators, clubs, players, and supporters, will actively participate in the project and provide timely feedback and input. Their engagement is crucial for gathering requirements, conducting user testing, and ensuring the app meets their needs and expectations.

    \item Access to Relevant Data: It is assumed that the necessary data, such as player profiles, fixture schedules, and league rules, will be available in a structured format for integration into the app. This data will serve as the foundation for various functionalities and reporting features within the app.

    \item Compliance with Applicable Regulations: It is assumed that the project will adhere to relevant laws, regulations, and data protection requirements. This includes ensuring compliance with data privacy regulations and obtaining necessary permissions or consents for the collection, storage, and processing of personal data within the app.

    \item Reliable Third-Party Integrations: It is assumed that any third-party services, APIs, or integrations required by the app will be available and reliable. This includes services for payment processing, communication tools, and other external dependencies. Thorough research and evaluation will be conducted to select trusted and stable partners for seamless integration.

    \item User Training and Adoption: It is assumed that appropriate training and support mechanisms will be in place to facilitate user onboarding and adoption of the app. Training materials, user documentation, and intuitive user interfaces will be developed to ensure users can effectively utilize the app's features and functionalities.

    \item Continuous Improvement and Maintenance: It is assumed that post-launch, there will be ongoing efforts to monitor and improve the app based on user feedback, evolving requirements, and technological advancements. Regular maintenance, bug fixing, and performance optimization will be undertaken to ensure the app remains reliable, secure, and aligned with the changing needs of the football league.
\end{itemize}
These assumptions provide a foundation for project planning and decision-making. It is important to regularly validate these assumptions throughout the project lifecycle and adapt as necessary to ensure the successful development and implementation of the Football League Management App.
\section{Vision}
The vision for the Football League Management App is to provide a cutting-edge, user-friendly software solution that revolutionizes the management of football leagues. The app aims to streamline administrative tasks, improve communication, and enhance the overall experience for league administrators, clubs, players, and supporters. It will serve as a comprehensive platform for managing all aspects of a football league, from club and fixture management to result tracking and financial administration. The key elements of the vision are outlined below:
\subsection{Vision Statement}
"To revolutionize the landscape of football league management through the development of a highly sophisticated and intuitive web-based application. Our vision is to empower league administrators, clubs, and players with an all-encompassing software solution that seamlessly integrates every facet of league management, fostering transparency, fair play, collaboration, and data-driven decision-making. By leveraging cutting-edge technologies and intuitive user interfaces, our goal is to create an unparalleled user experience that maximizes efficiency, elevates league standards, and transforms the way football leagues are managed and enjoyed."
\subsection{ List of features}
\subsubsection{User Registration and Login}
\begin{itemize}
    \item Secure registration process for league administrators, clubs, and players.
    \item User-friendly login functionality with role-based access control.
\end{itemize}
\subsubsection{Club Management}
\begin{itemize}
    \item Comprehensive tools for managing clubs, including creation, modification, and deletion.
    \item Ability to update club profiles, including team rosters, contact information, and logos.
    \item Customizable club attributes to accommodate specific league requirements.
\end{itemize}

\subsubsection{Fixture Management}
\begin{itemize}
    \item Intuitive interface for scheduling, updating, and canceling fixtures.
    \item Flexibility to define fixture details such as date, time, venue, and participating clubs.
    \item Automated fixture generation based on predefined rules and constraints.
\end{itemize}

\subsubsection{Result and Ranking Management}
\begin{itemize}
    \item Efficient result recording and management functionality.
    \item Calculation of league standings and rankings based on user-defined scoring criteria.
    \item Real-time updates of league tables with comprehensive filtering and sorting options.
\end{itemize}

\subsubsection{Club Self-Management}
\begin{itemize}
    \item Empowering clubs to manage their own information within the app.
    \item Update club profiles, including contact details, social media links, and sponsor information.
    \item Self-service functionality for managing player registrations, transfers, and availability.
\end{itemize}

\subsubsection{Mobile Accessibility}
\begin{itemize}
    \item  Responsive design and compatibility with mobile devices for on-the-go access.
    \item Native mobile applications for iOS and Android platforms, ensuring a seamless user experience.
\end{itemize}

These features collectively aim to deliver a comprehensive and robust football league management solution, addressing the diverse needs of league administrators, clubs, players, and supporters while driving efficiency, collaboration, and overall league excellence.

\subsection{ Scope of the Release}
The initial release of the Football League Management App will focus on delivering a solid foundation of core functionalities essential for effective league management. This includes secure user registration and login capabilities, club management features for creating and updating club profiles, fixture management functionalities for scheduling and updating matches, result and ranking management functionalities for recording results and calculating standings. The scope of the initial release is designed to provide the necessary tools for league administrators, clubs, and players to efficiently manage the essential aspects of their football league. Additional advanced features and functionalities will be considered for future releases, based on user feedback and evolving needs, to further enhance the app's capabilities and user experience.

\subsection{ Features that are Not Included }
In the initial release, certain features will not be included but may be considered for future releases based on user feedback and evolving needs. The features not included in the initial release are:
\begin{itemize}
    \item Advanced scheduling algorithms for fixture generation.
    \item Advanced result management features, such as comprehensive statistics and match analysis.
    \item Advanced financial management, including detailed budgeting and payment integration.
    \item Advanced communication tools, such as in-app messaging and document sharing.
    \item Advanced club self-management functionalities, such as player transfer management and performance tracking.
\end{itemize}
These features will be evaluated for subsequent releases to enhance the overall functionality and user experience of the Football League Management App.


\section{Conclusion}
The Football League Management App represents a significant step forward in streamlining and enhancing the management of football leagues. By leveraging innovative technologies and user-centric design, the app aims to provide a comprehensive solution that empowers league administrators, clubs, and players with efficient tools to manage various aspects of league operations. With its user-friendly interface, mobile accessibility, and powerful features, the app aims to revolutionize league management, fostering transparency, collaboration, and data-driven decision-making.

By automating administrative tasks, improving communication, and providing valuable insights through advanced reporting and analytics, the app will save time, reduce manual errors, and enhance the overall league experience. It will enable league administrators to focus on strategic planning, clubs to effectively manage their resources, and players to enjoy a well-organized and competitive environment.

As we embark on this journey, we remain committed to incorporating user feedback, addressing evolving needs, and continually enhancing the app to meet the changing landscape of football league management. We believe that the Football League Management App will transform the way leagues are managed, fostering fair play, sportsmanship, and a thriving football community.


\end{document}
